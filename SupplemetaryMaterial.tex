\documentclass[10pt]{article}
\usepackage[paper=letterpaper,margin=2cm]{geometry}
\usepackage{amsmath}
\usepackage{amssymb}
\usepackage{amsfonts}
\usepackage{newtxtext, newtxmath}
\usepackage{enumitem}
\usepackage[colorlinks=true]{hyperref}


\title{Understanding the Free Energy Landscape of Phase Separation in Lipid Bilayers using Molecular Dynamics}
\author{Ashlin J. Poruthoor, Alan Grossfield}

\begin{document}
\maketitle

\section*{Simulation details}

\section*{Auxiliary Variables}

For each lipid species, $X$ in the system, we calculated the following, 

1. Number of $X$ clusters in the system under study.

2. Fraction of $X_i$ lipids in $X$ clusters

3. Fraction of $X_i$ lipids in $X$ core lipids.

4. Mean Silhouette Coefficient (MSC) of $X_i$ Clusters, as implemented in scikit-learn.

Silhouette Coefficient is a method used to evaluate the clustering done by any technique, especially if ground truth labels are unknown. 
Here, for a $X_i$ lipid in the cluster, mean intra-cluster distance (a) from other $X_i$ lipids in the cluster is found.  
Similarly, for a $X_i$ lipid in the cluster, the mean nearest-cluster distance (b) is also calculated.
While the former assesses the 'cohesion' of a given $X_i$ lipids with other $X_1$ lipids in a cluster, the latter assesses the 'separation' from the nearest cluster.
Thus, Silhouette Coefficient for a $X_i$ lipid, s, is defined as below,

\begin{equation}
\label{eq:SC}
\text{s} = \frac{b - a}{max(a,b)}
\end{equation}

The Mean Silhouette Coefficient of $X_i$ Clusters is given by the mean $s$ over all non-outlier $X_i$ lipids.
Here, we have omitted the MSC calculations for cases when there are no clusters or just one cluster detected by DBSCAN. 
MSC is bound between and -1 and 1.
A high positive value corresponds to well segregated dense clusters, while a low negative value implies that lipids are assigned to clusters incorrectly.  

\end{document}